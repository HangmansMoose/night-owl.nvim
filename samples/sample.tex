\documentclass{article}

\usepackage{amsmath}
\usepackage{graphicx}
\usepackage{hyperref}
\usepackage{listings}
\usepackage{caption}

\title{Example LaTeX Document}
\author{Foobar}
\date{\today}

\begin{document}

\maketitle

\begin{abstract}
This is an example LaTeX document. It demonstrates various features of LaTeX, including sections, mathematical expressions, tables, figures, and more. The variable names used are foobar, foo, bar, baz, and the string literals include "Lorem Ipsum" and "123".
\end{abstract}

\section{Introduction}
Welcome to this example document. This section introduces the content. "Lorem Ipsum" is commonly used as placeholder text. The number 123 is often used in examples.

\section{Mathematical Expressions}
LaTeX is great for typesetting mathematical expressions. Here are some examples:

\subsection{Inline Math}
Inline math uses single dollar signs: $E = mc^2$.

\subsection{Displayed Equations}
Displayed equations use double dollar signs or the \texttt{equation} environment:
\begin{equation}
\int_0^\infty e^{-x^2} dx = \frac{\sqrt{\pi}}{2}
\end{equation}

\section{Lists}
LaTeX supports both ordered and unordered lists.

\subsection{Unordered List}
\begin{itemize}
    \item Foo
    \item Bar
    \item Baz
\end{itemize}

\subsection{Ordered List}
\begin{enumerate}
    \item First item
    \item Second item
    \item Third item
\end{enumerate}

\section{Tables}
Tables can be created using the \texttt{tabular} environment.

\begin{table}[h]
\centering
\begin{tabular}{|c|c|c|}
\hline
Foo & Bar & Baz \\
\hline
123 & 456 & 789 \\
\hline
Lorem & Ipsum & Dolor \\
\hline
\end{tabular}
\caption{Example Table}
\label{tab:example}
\end{table}

\section{Figures}
Figures can be included using the \texttt{figure} environment.

\begin{figure}[h]
\centering
\includegraphics[width=0.5\textwidth]{example-image}
\caption{Example Image}
\label{fig:example}
\end{figure}

\section{Code Listings}
You can include code using the \texttt{listings} package.

\begin{lstlisting}[language=bash, caption={Example Bash Script}]
#!/bin/bash

# Variables
foo="Lorem Ipsum"
bar=123
baz=3.14

# Print variables
echo "foo: $foo"
echo "bar: $bar"
echo "baz: $baz"

# Conditional statement
if [ $bar -gt 100 ]; then
    echo "bar is greater than 100"
else
    echo "bar is not greater than 100"
fi

# Loop
for i in {1..5}; do
    echo "Iteration $i"
done
\end{lstlisting}

\section{Hyperlinks}
You can include hyperlinks using the \texttt{hyperref} package. Here is an example: \href{https://www.example.com}{Example Link}.

\section{Conclusion}
This document provided an overview of various LaTeX features. We used variable names such as foobar, foo, bar, and baz, and string literals like "Lorem Ipsum" and "123".

\end{document}

